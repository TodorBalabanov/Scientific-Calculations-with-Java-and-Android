\documentclass[ebook,12pt,oneside,openany]{memoir}
\usepackage[utf8x]{inputenc}
\usepackage[bulgarian]{babel}
\usepackage{url}
\usepackage{lipsum}
% Според изискванията на ИИКТ-БАН не бива да има номерация на страниците в ръкописа.
\usepackage{nopageno}

% Заглавие на книгата.
\title{Научни изчисления с Java и Android}

% Имена на авторите.
\author{Тодор Балабанов, Илиян Занкински, Петър Томов}

\begin{document}

\maketitle

% Заглавната страница не е с оформлението на останалата част от документа (няма номерация).
\thispagestyle{empty}

% Тук стои таблицата със съдържанието, което се генерира от названието на главите.
\newpage
\tableofcontents

% По този начин номерацията на подточките е с арабски цифри.
\renewcommand\thesection{\thechapter.\arabic{section}}
\renewcommand\thesubsection{\thesection.\arabic{subsection}}

\newpage
\addcontentsline{toc}{chapter}{Предговор}
\chapter*{Предговор}

Това учебно помагало е предназначено за ученици и студенти, които биха искали да се запознаят с възможностите за реализиране на изчисления в разпределена среда с използването на програмния език Java и мобилната платформа Android.

В съвременното ни ежедневие ние все по-често сме заобиколени от мобилни изчислителни устройства. Най-често това са мобилни телефони, таблети, часовници или други форми на wearables (миниатюрна електроника под формата на модни аксесоари или части от дрехи). До преди десетилетие този вид мобилни устройства бяха рядкост, а изчислителните им възможности бяха изключително ограничени. Тези два факта не позволяваха мобилните устройства да бъдат използвани за нещо повече освен основния принцип на употреба за който са създадени. С развитието на микроелектрониката и миниатюризацията на компонентите съставящи мобилните устройства техните възможности значително нараснаха за последното десетилетие. Това позволява върху този вид устройства да се извършват и допълнителни задачи, които не са били предвидени при първоначалното им проектиране. Паралелно с развитието на мобилните технологии бурен подем претърпяха и възможностите за мобилна комуникация като GSM, 3G, 4G, Wi-Fi, Bluetooth, NFC и други. Комбинацията между относително мощни мобилни изчислителни устройства и добре развита комуникационна среда открива безгранични възможности за приложение на мобилните устройства при извършването на допълнителни изчисления, в разпределена среда. 

В настоящето учебно помагало ще запознаем читателите с интересните възможности, които предлагат съвременните Android мобилни устройства, за постигането на резултати в научни изчислителни задачи, разпределяйки изчисленията върху физически отдалечени едно от друго устройства. 

\newpage
\chapter{Научни изчисления}

\section{Последователни изчисления}

\section{Паралелни изчисления}

\section{Грид изчисления}

\section{Изчисления в разпределена среда}

\section{Дарена изчислителна мощност}

\newpage
\addcontentsline{toc}{chapter}{Заключение}
\chapter*{Заключение}

Основна цел на настоящото учебно помагало бе представянето на възможностите, които съвременните Android мобилни устройства могат да предложат за извършване на научни изчисления в разпределена среда. Без да претендира за изчерпателност изложеният материал има за цел да провокира творческото мислене у читателя и да го вдъхнови за създаването на собствени авторски проекти с разгледаните технологии и засегнатите научни области.  

Авторите са благодарни на своите читатели за отделеното време и внимание, като горещо насърчават подаването на обратна връзка и споделянето на интересни мисли, идеи или предложения, на посочените за връзка контакти.

\end{document}