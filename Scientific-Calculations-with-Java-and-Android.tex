\documentclass[book,14pt,oneside,openany]{memoir}
\usepackage[utf8x]{inputenc}
\usepackage[bulgarian]{babel}
\usepackage{url}
\usepackage{lipsum}
% Според изискванията на ИИКТ-БАН не бива да има номерация на страниците в ръкописа.
\usepackage{nopageno}
\usepackage{shorttoc}
\usepackage{graphicx}
\usepackage{imakeidx}
% Добавя възможност за сензитивни хипер-връзки в самия документ.
% \usepackage{hyperref}
\usepackage{placeins}

% Заглавие на книгата.
\title{Научни изчисления с Java и Android}

% Имена на авторите.
\author{Тодор Балабанов, Илиян Занкински, Петър Томов}

% Автоматично създаване на азбучен указател.
\makeindex[columns=1, title=Азбучен указател, intoc]

\begin{document}

\maketitle

% Заглавната страница не е с оформлението на останалата част от документа (няма номерация).
\thispagestyle{empty}

% Тук стои таблицата със съдържанието, което се генерира от названието на главите.
\newpage
\shorttoc{Теми}{0}

% Тук стои таблицата със съдържанието, което се генерира от названието на главите и названието на секциите в тях.
\newpage
\tableofcontents

% По този начин номерацията на подточките е с арабски цифри.
\renewcommand\thesection{\thechapter.\arabic{section}}
\renewcommand\thesubsection{\thesection.\arabic{subsection}}

\newpage
\addcontentsline{toc}{chapter}{Предговор}
\chapter*{Предговор}

Това учебно помагало е предназначено за ученици и студенти, които биха искали да се запознаят с възможностите за реализиране на изчисления в разпределена среда\index{изчисления в разпределена среда} с използването на програмния език Java и мобилната платформа Android.

В съвременното ни ежедневие ние все по-често сме заобиколени от мобилни изчислителни устройства. Най-често това са мобилни телефони, таблети, часовници или други форми на wearables (миниатюрна електроника под формата на модни аксесоари или части от дрехи). До преди десетилетие този вид мобилни устройства бяха рядкост, а изчислителните им възможности бяха изключително ограничени. Тези два факта не позволяваха мобилните устройства да бъдат използвани за нещо повече освен основния принцип на употреба за който са създадени. С развитието на микроелектрониката и миниатюризацията на компонентите съставящи мобилните устройства техните възможности значително нараснаха за последното десетилетие. Това позволява върху този вид устройства да се извършват и допълнителни задачи, които не са били предвидени при първоначалното им проектиране. Паралелно с развитието на мобилните технологии бурен подем претърпяха и възможностите за мобилна комуникация като GSM, 3G, 4G, Wi-Fi, Bluetooth, NFC и други. Комбинацията между относително мощни мобилни изчислителни устройства и добре развита комуникационна среда открива безгранични възможности за приложение на мобилните устройства при извършването на допълнителни изчисления, в разпределена среда\index{изчисления в разпределена среда}. 

В настоящето учебно помагало ще запознаем читателите с интересните възможности, които предлагат съвременните Android мобилни устройства\index{мобилни устройства}, за постигането на резултати в научни изчислителни задачи, разпределяйки изчисленията върху физически отдалечени едно от друго устройства. 

\newpage
\chapter{Научни изчисления}

Още в зората на съвременната изчислителна техника най-съществените пресмятания са били с научна насоченост и военно дело. Този факт не се е променил значително за последните десетилетия. Дори в наши дни най-сериозните изчислителни ресурси са насочени в областта на науката. Това дава основание да обърнем значително внимание на начините по които можем да изпълняваме научни изчисления дори и върху изчислителни устройства, чието основно предназначение не е с научна цел. 

\section{Последователно програмиране}

При последователното програмиране\index{последователното програмиране} всяка изчислителна инструкция следва всички предходни. В зората на изчислителната техника пресмятанията са извършвани по този начин. Дори в наши дни значителна част от алгоритмите се изпълняват само последователно, тъй като входните данни за всяка инструкция зависят от изходните данни на предходните инструкции. Последователните алгоритми не подлежат на декомпозиране и поради тази причина са неприложими за паралелни пресмятания. 

\begin{figure}[h!]
  \centering
  \includegraphics[width=1.0\linewidth]{./images/pic0001.png}
  \caption{Сравнение между последователни пресмятания и паралелни пресмятания.}
\label{fig:pic0001}
\end{figure}

\section{Паралелно програмиране}

При паралелното програмиране основно се говоир за две разновидности - конкурентни пресмятания\index{конкурентни пресмятания} и паралелни пресмятания\index{паралелни пресмятания}. Конкурентните пресмятания са в среда където група от задачи могат да се пресметнат едновременно, без да има значение от реда на пресмятане. В същото време, паралелните пресмятания се отнасят за едновременно пресмятане на отделни задачи, върху отделни процесори. В този контекст всички паралелни пресмятания са конкурентни пресмятания, но не и обратното (Фиг. \ref{fig:pic0001}). 

\begin{figure}[h!]
  \centering
  \includegraphics[width=1.0\linewidth]{./images/pic0002.png}
  \caption{Сравнение между конкуретни пресмятания и паралелни пресмятания.}
\label{fig:pic0002}
\end{figure}

\section{Супер компютри и грид изчисления}

Когато паралелните алгоритми се изпълняват на изчислителни машини с множество процесори и/или множество ядра на процесорите, този вид изчисления се определят като супер компютърни\index{супер компютърни изчисления} (supercomputing). Същественото при този вид пресмятания е, че се използва много бърз вътрешна шина (понякога оптична) и споделена оперативна памет. За разлика от супер компютрите\index{супер компютърни изчисления} грид изчисленията\index{грид изчисленията} се осъществяват на множество машини, свързани в обща мрежа, но работещи автономно, без да споделят обща памет. При грид системите отделните изчислителни машини могат да са териториално отдалечени една от друга. Съществено е да се отбележи, че и при супер компютрите\index{супер компютърни изчисления} собственикът на системата има пълен контрол над нея. Това може малко да се различава за грид системите, ако към грида са включени компютри под чужд контрол. 

\section{Изчисления в разпределена среда}

Преходът от грид системите към системи за изчисления в разпределена среда\index{изчисления в разпределена среда} се състои в това, че изчислителните машини в разпределената среда са абсолютно автономни. Тези машини не споделят общи ресурси, като процесор или оперативна памет. Много характерно е в разпределената среда контролът над изчислителните машини да не е от страната на организиращия изчисленията. Това води до два основни проблема - ненадеждна (често и твърде бавна) комуникация, липса на гаранция за коректност на пресмятанията (манипулации от страна на притежаващия изчислителните ресурси). Също така, при грид системите често се наблюдава хомогенност на изчислителните ресурси по отношение на хардуерни конфигурации и операционна система, докато в разпределената среда изчислителните машини са основно хетерогенни, което може да води до големи разлики в хардуера и операционните системи. 

\section{Дарена изчислителна мощност}

За решаването на някои по-мащабни научни проблеми изчислителната мощност и/или финансовите ресурси често са недостатъчни. В такива ситуации не малко научни институции прибягват до така наречените дарени изчислителни ресурси\index{дарени изчислителни ресурси} в разпределена среда\index{изчисления в разпределена среда}. Един от най-изчерпателните списъци с проекти от този вид може да бъде открит в уеб сайта Distributed Computing Info \cite{dcinfo}. Съществено е да се отбележи, че не всеки изчислителен проблем е подходящ за решение в разпределена среда\index{изчисления в разпределена среда} с дарена изчислителна мощ. На първо място проблемът трябва да подлежи на декомпозиране, така че отделни части от него да се пресмятат едновременно. Второто важно нещо е да не е от съществено значение в кой момент от времето и в какъв ред ще бъдат получени пресметнатите резултати. И третото съществено нещо е да е наличен механизъм за проверка на достоверността от пресмятанията, тъй като изчисленията се извършват на машини с различна хардуерна конфигурация и различни операционни системи, а освен това са възможни манипулации от страна на хората притежаващи тези машини. Най-известният проект за дарена изчислителна мощ е SETI@home \cite{shuch}, като неговата цел е да търси сигнали от космоса, които да са създадени от интелигентни форми наживот.

Когато изчисленията се извършват на мобилни устройства, то разпределената среда се превръща в мобилна среда за разпределени изчисления\index{мобилна среда за разпределени изчисления}. В останалата част от това учебно помагало ще бъде представено точно изграждането на система за извършване на разпределени изчисления върху мобилни устройства. 

\newpage
\chapter{Евристични алгоритми}

Евристичните алгоритми са подход за решаване на изчислителни проблеми, основаващи се на опита и интуицията. Този вид алгоритми не гарантират оптимално решение за поставената задача. Евристичните алгоритми намират широко приложение при задачи, които трудно се поддават на точни числени методи или аналитични решения. Макар и да не могат да предложат оптимално решение, евристичните алгоритми\index{евристични алгоритми} често водят до достатъчно приемливи в практиката, близки до оптималното решения. Благодарение на своята не детерминистична природа, евристичните алгоритми\index{евристични алгоритми} са изключително подходящи за реализация в супер компютърни изчисления\index{супер компютърни изчисления}. Две изключително популярни евристики\index{евристики} са изкуствените невронни мрежи\index{изкуствени невронни мрежи} и генетичните алгоритми\index{генетични алгоритми}. Те са обект на особена популярност през последните две десетилетия и това дава основание да бъдат заложени в основното изложение на настоящото учебно помагало. Сами по себе си евристиките са безполезни, ако не бъдат приложени върху достатъчно сложна изчислителна задача. Точно такава сложност предлагат задачите за прогнозиране. Прогнозирането е залегнало в множество дейности от човешкото ежедневие, като започнем от средните дневни температури и стигнем до потреблението на определени стоки и услуги. Под една или друга форма почти всяка човешка дейност бива остойностена в термините на финансовите ресурси. Този факт дава основание да бъдат разгледани точно прогнози за промяната в цените на различни финансови инструменти. Отчитането на промяната в цената, на определени интервали (равни или неравни), води до представяне на информацията във времеви ред\index{времеви редове}. При времевите редове\index{времеви редове} по абсцисната ос се означава времето, а по ординатната ос стойността на измерваната величина (в конкретния случай цената).

\section{Финансови времеви редове}

Времевите редове\index{времеви редове} са серия от замервания извършени в последователен ред във времето. Практически се получава последователност от дискретни стойности. Времевите редове могат да са съставени от стойности на равни интервали или стойности на произволни интервали. Често в практиката се случва да има липсващи замервания, което води до определени усложнения при анализирането. При финансовите времеви редове основно се използват равни интервали на отчитане и рядко има липсващи стойности при отчитането. Всяко едно отчитане при финансовите времеви редове се отличава с група стойности, характеризиращи времевия интервал за който се отнася, а именно – начална стойност за интервала, най-висока постигната стойност за интервала, най-ниска стойност постигната за интервала и крайна стройност за интервала. 

\begin{figure}[h!]
  \centering
  \includegraphics[width=1.0\linewidth]{./images/pic0003.png}
  \caption{Отношението евро/долар в рамките на един час.}
\label{fig:pic0003}
\end{figure}

На Фиг. \ref{fig:pic0003} тази информация се обозначава с вертикални черти, като долният край на вертикалната черта символизира най-ниското ниво, горният край най-високото ниво, а двете странични чертички маркират нивото на отваряне (отляво) и нивото на затваряне (от дясно). Анализирането на времевите редове\index{времеви редове} може да предостави съществена информация за лицата вземащи решения. Когато става въпрос за времеви редове във финансовата област основна цел на анализа е предлагането на прогноза. Една от най-често използваните форми на анализ е напасването на крива (curve fitting). От математическа гледна точка задачата представлява построяване на крива която да премине максимално близо до предварително определени точки. От математиката е добре известно, че през краен брой точки могат да се построят безкрайно на брой преминаващи криви. За да има смисъл от напасването кривата трябва да притежава определени свойства, като изглаждане, добра интерполация и екстраполация. 

Условното разделяне на финансовия времеви ред\index{финансови времеви редове} на две половини (минало и бъдеще) дава изключително добра възможност за представяне на задачата по прогнозиране в термините на изкуствените невронни мрежи\index{изкуствени невронни мрежи}. Мащабираната информация от миналия период (зелено на графиката) се подава към входния слой на изкуствената невронна мрежа, а прогнозата се получава в изходния слой на изкуствената невронна мрежа и се подлага да обратно мащабиране (червеното на графиката). Така декомпозиран времевият ред\index{времеви редове} се използва при фазата за обучение на изкуствената невронна мрежа. За работната фаза на мрежата за прогноза на изкуствената невронна мрежа се подават последните измерени стойности, без да има яснота какви са бъдещите измервания. 

\section{Изкуствени невронни мрежи}

Изкуствените невронни мрежи\index{изкуствени невронни мрежи} са се появили в следствие на опитите за изграждане на математически модел за биологичните нервни системи. Този вид системи се научават (прогресивно подобряват възможностите си) с изследването на примерни данни. Приложението им е основно при задачи за които традиционните алгоритми не дават приемливи резултати. Най-широко приложение изкуствените невронни мрежи намират в задачи за класификация. Когато става въпрос за финансови времеви редове са възможни две събития – повишаване на стойността или понижаване на стойността. Според информацията в отминалите периоди време, изкуствената невронна мрежа\index{изкуствени невронни мрежи} може да раздели данните в два основни класа – данни показващи промяна към повишение или данни показващи промяна към понижение. 

\begin{figure}[h!]
  \centering
  \includegraphics[height=0.25\pdfpageheight]{./images/pic0004.png}
  \caption{Трислойна изкуствена невронна мрежа.}
\label{fig:pic0004}
\end{figure}

Структурата на класическите изкуствени невронни мрежи\index{изкуствени невронни мрежи} се състои от възли, наречени изкуствени неврони и връзки, наречени тегла (Фиг. \ref{fig:pic0004}). Връзките между невроните служат за предаване на сигнал от един неврон към друг. Невроните които получават сигнали ги обработват по предварително заложено правило и след това могат на свой ред да ги разпространят към други неврони. В класическият случай невроните имат стойност (обикновено това е реално число). Връзките между невроните също са представени със стойност и според правилото за обучение точно тази стойност подлежи на промяна, така че мрежата да заучава необходимата информация. При най-използваните изкуствени невронни мрежи\index{изкуствени невронни мрежи} невроните са организирани в слоеве. Сигналите при този вид мрежи се разпространяват от входния слой към изходния слой, като преминават междинните слоеве. Това разпространение на сигналите се нарича пас в права посока и е характерно за режима на експлоатация. Освен режим на употреба изкуствените невронни мрежи\index{изкуствени невронни мрежи} работят и във втори режим, наречен режим на обучение. За последните няколко десетилетия са предложени множество начини за обучение на изкуствени невронни мрежи, като най-значими резултати се получават при използването на алгоритъма за обратно разпространение на грешката\index{обратно разпространение на грешката}. Обратното разпространение на грешката представлява точен числен метод от групата на градиентните методи, който разчита на грешката, която мрежата допуска при изпълнението на обучаващите примери. Тази допусната грешка се установява в изходния слой и след това се разпространява обратно по предходните слоеве, от където идва и названието на метода. 

Според своята линейна природа, обратното разпространение на грешката\index{обратно разпространение на грешката} трудно се поддава на реализация в термините на паралелното програмиране. Ако се погледне на теглата в една изкуствена невронна мрежа като на многомерно безкрайно пространство от реални числа, то намирането на стойности за теглата е математическа оптимизационна задача. Точните числени методи дават добри резултати при пространства с относително малка размерност, но срещат сериозни затруднения при по-големите размерности. Точно противоположно на точните числени методи, евристичните алгоритми предлагат приемливи решения, в разумно време. Предимство е не само ефективността, но и значително по-големите възможности евристичните алгоритми да се реализират в термините на паралелното програмиране. 

Класическите неврони реализират трансферна функция и активационна функция. Най-често използваната трансферна функция е линейната, която представлява сума от умножение на входните сигнали по теглата, които ги доставят. Най-често използваните активационни функции са сигмоидната функция и хиперболичния тангенс. Активационната функция има основна роля за нормиране на изходния сигнал. Тази нормализация е необходима тъй като за трансферната функция, при различните неврони, постъпват различен брой сигнали и без нормализация това би направило изходните сигнали несъизмерими. При градиентните точни числени методи има изискване активационната функция да бъде диференцируема, нещо което не е необходимо при евристичните алгоритми. Често използваната линейна трансферна функция има нужда от използване на допълнително събираемо наречено „отместване“ (bias). От формална гледна точка, отместването може да се интерпретира като тегло свързващо неврона с друг неврон, който емитира единичен сигнал. В практиката за всеки слой е прието да се отделя неврон емитиращ единичен сигнал. Само в изходния слой е безсмислено такъв неврон да има, тъй като той не получава входни сигнали, а емитирането на постоянна единица в изхода не носи смислена информация за функционирането на мрежата. 

От математическа гледна точка класическите невронни мрежи могат да се представят с вектор (стойностите на невроните) и матрица (стойностите на теглата между невроните). Разпространението на сигналите от входа към изхода в този случай би бил умножение на вектор с число. В настоящото учебно помагало на класическа трислойна мрежа ще се подават мащабираните стойности от изминалите времеви периоди, а на изхода ще се очаква прогнозна стойност за бъдещи времеви интервали. Тази концепция е малко по-сложна от идеята за проста квалификация от вида повишение/понижение, но е значително по-информативна, защото се очаква да дава представа за един по-дълъг времеви интервал в бъдещето. След успешно прогнозиране в коя посока ще се промени цената от съществено значение става и въпросът колко дълго ще продължи този спад/повишение. 

\section{Генетични алгоритми}

Генетичните алгоритми представляват глобална оптимизационна евристика\index{глобална оптимизационна евристика} вдъхновена от идеите за биологичната еволюция. Генетичните алгоритми са подмножество на класовете популационни алгоритми и еволюционни алгоритми. Основното си приложение генетичните алгоритми\index{генетични алгоритми} намират при задачи с голяма по размерност пространство на решенията. Често при такива задачи класическите точни числени методи не могат да предложат решение в приемливо време. За да се приложи генетичен алгоритъм решението на съответната задача трябва да се представи под формата на хромозома (индивид) в обща популация от решения. След това, чрез прилагане на основните операции по селекция\index{селекция в генетичен алгоритъм}, кръстосване\index{кръстосване в генетичен алгоритъм} и мутация\index{мутация в генетичен алгоритъм} (Фиг. \ref{fig:pic0005}), отделните индивиди (решения) следва да бъдат подобрявани.

\begin{figure}[h!]
  \centering
  \includegraphics[width=1.0\linewidth]{./images/pic0005.png}
  \caption{Трите основни операции в генетичните алгоритми.}
\label{fig:pic0005}
\end{figure}

При класическия процес оптимизацията започва от случайно генерирани индивиди, Това не е задължително особено когато става въпрос за хибридни реализации в които генетичният алгоритъм е поддържаща оптимизация. При такива ситуации началната популация може да бъде получена в резултат на друга оптимизация или в резултат на човешка подредба. След началната фаза оптимизацията протича итеративно и приключва според предварително определени критерии за край. Най-често се използва предварително дефиниран брой поколения или брой поколения които не водят до подобрение в намерените решения, но също така е възможно да се дефинира и интервал астрономическо време. 

От основна важност за успешната работа с генетичните алгоритми\index{генетични алгоритми} е определянето на целева функция (жизнена функция\index{жизнена функция} на индивида) която еднозначно да определя качеството на полученото решение. На база различната жизненост която индивидите в популацията притежават се взема стохастично решение кои индивиди да участват в създаването на бъдещото поколение и кои не. В множество реализации на генетични алгоритми се прилага правило на елита, така че най-доброто открито решение да достигне края на оптимизационния процес. В същото време правилото на елита крие риск от израждане на популацията, така че всичките решения в нея да клонят към съхранения елит.  

Фактът, че генетичните алгоритми са организирани на принципа на популацията от индивиди ясно подсказва идеалната възможност оптимизационния процес да се организира не в една глобална популация, а в множество различни локални популации които да съществуват на различни изчислителни машини. Това от своя страна би дало възможност за реализиране на миграционни процеси, точно както това се наблюдава при естествените биологични видове. 

\section{Прогнозиране в разпределена среда}

Гъвкавите възможности на изкуствените невронни мрежи\index{изкуствени невронни мрежи} да изграждат функционална зависимост между входни и изходни данни ги прави идеален кандидат за прогнозираща система. Ако се приеме, че измерените стойности във времевия ред са точки в двуизмерно пространство, то задачата за прогнозиране може да се представи като задача за прекарване на крива през N точки (curve fitting). Ако образно се оприличи изкуствената невронна мрежа на полином, то теглата й биха представлявали коефициенти в полинома. Стойностите които изкуствената невронна мрежа\index{изкуствени невронни мрежи} генерира на изхода си за бъдещи моменти от времето представляват своеобразна екстраполация\index{екстраполация} според апроксимираната крива. Тъй като класическите многослойни невронни мрежи работят с входни сигнали между 0.0 и 1.0 или -1.0 и +1.0, то информацията от времевия ред трябва да бъде мащабирана в съответния работен интервал на мрежата. Стойностите на времевия ред условно се разделят на минали (зелените Фиг. \ref{fig:pic0003}) и бъдещи (червените Фиг. \ref{fig:pic0003}). Изходната (прогнозна) информация на изкуствената невронна мрежа след това се мащабира обратно към оригиналните интервали на времевия ред.

\begin{figure}[h!]
  \centering
  \includegraphics[width=1.0\linewidth]{./images/pic0006.png}
  \caption{Продуктовата линия TradingSolutions използваща изкуствени невронни мрежи и генетични алгоритми за прогнозиране на Forex финансови инструменти.}
\label{fig:pic0006}
\end{figure}

От чисто практическа гледна точка, използването на изкуствени невронни мрежи и генетични алгоритми се е доказало като удачен подхода за прогнозиране на финансови инструменти, което ясно се вижда в продуктовата линия на TradingSolutions (Фиг. \ref{fig:pic0006}). Почти петнадесет години продуктовата линия на TradingSolutions доставяше системи за подпомагане вземането на решения\index{системи за подпомагане вземането на решения}. По настояще тази продуктова линия е придобита от компанията nDimensional, Inc.

\begin{figure}[h!]
  \centering
  \includegraphics[height=0.4\pdfpageheight]{./images/pic0007.png}
  \caption{Системата MoneyBee за финансово прогнозиране в разпределена среда.}
\label{fig:pic0007}
\end{figure}

Възможността да се прогнозират цените на финансови инструменти винаги е привличала не само индустрията, но и научните среди. Едно от най-забележителните постижения в тази насока е проектът MoneyBee (Фиг. \ref{fig:pic0007}). Макар и вече да не съществува, този проект предлагаше възможност за изчисляване на прогнози с помощта на дарена изчислителна мощност\index{дарени изчислителни ресурси} в разпределена среда\index{изчисления в разпределена среда} \cite{bohn}. Същинските прогнози се пресмятаха, с помощта на изкуствени невронни мрежи, върху изчислителните машини на потребителите в периоди когато натоварването на машините е ниско и се активира програмата за защита на монитора (screensaver). Преди масовото навлизане на мобилните устройства честа практика при проектите за дарена изчислителна мощност, с цел пресмятане в разпределена среда\index{изчисления в разпределена среда}, основен подход бе извършване на пресмятанията в специално създадена за целта програма за предпазване на монитора. След навлизането на катодно лъчевите тръби при настолните компютри се появява ефект от увреждане на монитора, ако върху него продължително се визуализира статична картина. Решаването на този проблем се оказва най-удачно с помощта на операционната система, която да установи период в който потребителя не използва изчислителната машина и да активира софтуерна програма за предпазване на монитора. Развитието на мониторите с течни кристали постепенно изведе от употреба мониторите с катодно-лъчеви тръби и използването на програми за предпазване на монитора изгуби своето първоначално предназначение. Въпреки това този вид софтуерни решения останаха в употреба и основно служат за повишаване на информационната сигурност, като не само дават естетическа визуализация, но и привеждат работната сесия на потребителя в заключено състояние, което възпрепятства използването на компютърната система от неауторизирани потребители. Наличието на изчислителни ресурси, които са налични, но неефективно използвани дава основание на множество учени да разработят системи за отдалечено пресмятане в разпределена среда\index{изчисления в разпределена среда}, точно под формата на програми за предпазване на монитора, които да се възползват от дарените потребителски, изчислителни ресурси. 

\section{Фонови пресмятания върху мобилни устройства}

Съществува концептуална разлика в начина по който потребителите използват настолните си компютри и мобилните устройства. На първо място, настолният компютър бива пускан и спиран според нуждите на потребителя, докато най-често мобилните устройства са в непрекъснат режим на употреба. Това води до основната разлика, че мобилните устройства не изпадат в режим на занижена употреба, но също така имат режими за употреба при изключителна важност (примерно телефонно обаждане с висок приоритет). Втората фундаментална разлика се състои във факта, че основен похват за пестене на електрическа енергия, доставяна предимно от батерии при мобилните устройства, е динамичното изгасяне на екрана. Тази стратегия за пестене на енергия кардинално отменя концепцията за програма предпазваща монитора. За да се реализира ефективно система за разпределени пресмятания върху мобилни устройства е много по-удачно да се използва технологията за активен десктоп, отколкото да се залага да идеята за програма предпазваща монитора. Точно тази идея е развита в настоящото учебно помагало. 

\newpage
\chapter{Софтуерна архитактура}

Разработката на софтуер е в областта на инженерните науки, тъй като продуктът се създава по принципите за изграждане на конструкции (в случая софтуерни). При стартирането на нов софтуерен проект трябва да се вземат редица решения, според заданието на потребителя. В настоящата разработка целта е софтуерно решение което извършва изчисления от страната на клиентски мобилни устройства. Задачите за пресмятане се възлагат от сървър и получените пресметнати резултати се получават обратно на същата машина. От страната на клиента се получават стойности за цена на валути под формата на времеви ред. Сървърът изпраща на клиента също информацията за топология на изкуствена невронна мрежа. Клиента от своя страна използва котировките на валутите и информацията за изкуствената невронна мрежа за да извършва пресмятанията необходими за обучението на изкуствената невронна мрежа. Процесът по обучението на изкуствената неверонна мрежа\index{изкуствени невронни мрежи} се извършва с помощта на генетични алгоритми\index{генетични алгоритми}, които имат за цел търсене на възможно по-оптимални стойности за теглата на мрежата. Клиентското приложение също поема отговорностите за визуализация на процеса по обучение и визуализация на постигнатите прогнози. Така представена системата съвсем естествено води към избора на „клиент-сървър“ софтуерна архитектура. 

\section{Избор на развойни средства}

В съвременната софтуерна индустрия има голям избор от развойни средства\index{развойни средства} за различните нужди на софтуерните разработчици. Една част от развойните средства са комерсиални, докато друга част са инструменти с отворен код. В настоящата разработка акцентът основно пада върху развойни средства с отворен код, тъй като минимизирането на разходите за производство е основен стремеж с цел постигане на икономическа ефективност. Изборът на развойни инструменти е задача от областта на мококритериалния анализ и основно се характеризира с наличието на множество критерии, които често са противоречиви. 

\subsection{От страна на сървъра}

За уеб базирани сървър решения най-популярни са технологиите JSP, ASP, PHP и Node.js. Тъй като ASP е комерсиална технология на фирмата Microsoft тя не представлява интерес за настоящата разработка. JSP e технология на фирмата Oracle която е с отворен код и дава възможност за изграждане на стабилни корпоративни решения. Недостатък на JSP е нуждата от по-сериозни софтуерни и хардуерни ресурси по отношение на хостинга. Node.js е технология, която набира все по-голяма популярност, но все още не е достигнала достатъчно ниво на „зрялост“, като същевременно също изисква повече софтуерни и хардуерни ресурси от страна на хостинга. По отношение на PHP, технологията е с тясно предназначение и има едно от най-високите нива на „зрялост“. В същото време разходите за хостинг при PHP са едни от най-ниските, което прави избора на тази технология изключително икономически ефективно. Задачата на уеб базираната сървър технология е да служи като посредник между мрежата и системата за управление на бази от данните. От страна на сървъра най-рационално е данните да се съхраняват в релационна база данни. Съществуват множество решения които да бъдат приложени в тази част на системата, като най-популярните са: Oracle, MS SQL Server, PostgreSQL и MySQL. Oracle намира своето приложение в корпоративния сегмент и е свързан със значителни финансови разходи, което го прави неприемлив за настоящата разработка. MS SQL Server е система алтернативна на Oracle в корпоративния сегмент и също е свързана със значителни финансови разходи. PostgreSQL е система с отворен код, която е съизмерима с технологичните възможности на Oracle и е потенциално добър кандидат за ниско бюджетни разработки. В настоящата разработка PostgreSQL е избегнат поради своята ненужна сложност, при едно относително просто софтуерно решение. Изборът пада върху MySQL, тъй като системата е максимално опростена добре наложена сред потребителите и поддържа се от фирмата Oracle. Освен всичко изброено, MySQL има добра поддръжка при хостинг доставчиците и е икономически най-ефективният избор. 

\subsection{От страна на клиента}

При „умните“ мобилни устройства най-разпространените операционни системи са Android, iOS и Windows Phone. По настояще фирмата Microsoft преустанови развитието на своята операционна система Windows Phone, което моментално води до отхвърлянето й за настоящата разработка. Инвестицията за разработка на приложения под iOS на фирмата Apple води до отпадането на тази платформа за нуждите на настоящата разработка. Към разходите за разработка може да се добави и факта, че програмирането за iOS се извършва на два езика Objective-C и Swift, който имат относително малка популярност в Източна Европа. Най-много „умни“ мобилни устройства в световен мащаб се използват с операционната система Android. Android е с отворен код, поддържа се основно от компанията Google и позволява разработка с значително по-ниски финансови разходи, спрямо конкурентите си. Приложенията за Android основно се разработват на езика Java, който по настояще е един от най-широко използваните програмни езици и се характеризира с много висока степен на „зрялост“. 

\subsection{За комуникация между сървъра и клиента}

Съществуват множество възможности за изграждането на комуникацията между сървъра и клиента. В най-суров вид информацията може да се предава като серия байтове (plane text), което води до множество затруднения при получаването и последващата й обработка. Широко използвана алтернатива е тагиращия език XML. При този вариант информацията бива „пакетирана“ в серия от тагове, които дават определена структура и семантика. Основният замисъл при проектирането на XML е бил създаването на структурирани документи, както от хора, така и от машини. Поради тази причина XML има една по-голяма експресивност в сравнение с неговата алтернатива JSON. JSON е максимално опростен тагиращ език за структурирано представяне на информацията, който води своето начало от обектите в програмния език JavaScript. За разлика от XML, JSON има основно предназначение за обмяна на структурирана информация между машини, а не толкова между хора и машини. Поради всичко изброено, в настоящата разработка изборът пада върху JSON като основа за изграждането на комуникационния протокол между сървъра и неговите клиенти. Тъй като от страната на сървъра се предвижда уеб базирано решение, то JSON базирания протокол ще протича в комуникационни сесии на HTTP протокола. В зората на уеб страниците е разработен протокола HTTP, стъпващ на TCP/IP, за ефективно предаване на информация между уеб сървърите и уеб браузърите. HTTP протоколът е добре наложен и с добра поддръжка в световен мащаб. Една важна негова характеристика е, че при този протокол комуникацията е разделена на заявки и отговори, без да се поддържа постоянна комуникационна линия. 

\section{Компоненти на системата}

След направеният кратък обзор на технологии и развойни средства изборът е за направата на „клиент-съръвр“ система. От страна на сървъра се разполагат модули за съхранение на данните (MySQL система за управление на бази от данни) и за комуникация (PHP уеб скриптове). Уеб сървърът комуникира с клиентите на база JSON/HTTP комуникационен протокол. Android мобилни устройства правят уеб заявки, извършват изчисленията и връщат резултата до сървъра. Тъй като се разчита на дарена изчислителна мощност\index{дарени изчислителни ресурси} е нужно да се избере подходящ начин за използване на мобилното устройство, без това да нарушава основните му функции и без да пречи на потребителите. Със своята технология Active Wallpaper, Android предлага идеална възможност за цените на настоящата разработка. Активният десктоп представлява изображение, което се изрисува зад всички основни графични компоненти от графичния потребителски интерфейс на Android. По-същественото е, че активния десктоп е Java приложение, което работи в постоянен фонов режим и може да изпълнява определени кратки задачи, когато устройството не е високо натоварено. 

\begin{figure}[h!]
  \centering
  \includegraphics[height=0.25\pdfpageheight]{./images/pic0008.png}
  \caption{Уеб базирана трислойна софтуерна архитектура.}
\label{fig:pic0008}
\end{figure}

При реализацията на настоящата система за изчисления в разпределена среда\index{изчисления в разпределена среда} изборът пада върху класическа трислойна софтуерна архитектура. Същият подход за три слоя се прилага и при реализацията на мобилното приложение, където SQLite локално съхранява данните над които се работи, Java обектно-ориентиран код извършва изчисленията, а Android базиран графичен потребителски интерфейс поема отговорността за визуализацията на процеса пред потребителя. 

\subsection{Подход за разработка}

\begin{figure}[h!]
  \centering
  \includegraphics[width=1.0\linewidth]{./images/pic0009.png}
  \caption{Подходи за софтуерна разработка.}
\label{fig:pic0009}
\end{figure}

Ако приемем, че визуализацията е най-горният слой, а базата данни най-долният слой на една трислойна софтуерна архитектура, то има два основни подхода за изработване на софтуерната система (Фиг. \ref{fig:pic0009}). При първия подход първоначално се разработва визуалния интерфейс, след това слоя на работната логика и накрая базата данни. Този подход се нарича „top-down“и е полезен когато се анализира софтуерно задание при което вече има в употреба множество хартиени документи (първоизточници на работните екрани. Този подход също е удобен при проекти с малък размер, където базата данни е относително опростена. Вторият много популярен подход е когато се започне с проектирането на базата данни, след това работната логика която обработва данните и едва накрая екраните визуализиращи информацията. Този подход се нарича „bottom-up“ и е най-подходящ когато се разработва сложно софтуерно решение, което се очаква да работи с големи обеми данни и множество различни структури на информацията. Акцентът в настоящата разработка е върху изчисленията които мобилните устройства извършват, а не толкова към създаването на голям масив от данни. Поради тази причина изборът в случая пада върху подхода „top-down“. 

\section{Лиценз и хранилище за проекта}

При съвременните софтуерни проекти с отворен код са от значение две неща – юридическият лиценз\index{софтуерни лицензи} под който съществува проекта и публичното хранилище\index{хранилища за програмен код} в което е разположен програмният текст. 

\subsection{Лиценз}

От съществено значение е изборът на правилен софтуерен лиценз, когато се разработва софтуерен проект предвиден да бъде публично достъпен. Съществуват множество възможности, като някои от най-популярните са - BSD License, MIT license, Mozilla Public License и GNU General Public License. За нуждите на настоящата разработка е предпочетен GNU General Public License v3, защото този лиценз е най-предпазващ за създателя на софтуерния продукт. В най-общи линии GPL3 позволява - комерсиална употреба, модификации, разпространение, включване в патенти и употреба за лични нужди. Лицензът съпровожда изключително ограничена отговорност за създателите на продукта и абсолютно никаква гаранция за употребата му от страна на потребителите. Лицензът също налага и серия ограничения – задължително включване на текст за авторските права на създателите, списък на извършените промени, не позволява закриване на кода и задължава всяко надграждане на продукта да бъде под същия лиценз. Със своята протекционистка природа GPL е един от лицензите дал най-силен тласък в развитието на продукти с отворен код, което е достатъчна причина да бъде избран за разработки без ясно изразена комерсиална насоченост. 

\subsection{Хранилище за програмен код}

Прието е всеки проект да има название, което особено важи в света на софтуера с отворен код. За настоящата разработка е избрано името VitDisComp. Точно това название е избрано, тъй като разработката ще се възползва от наличното сървър решение в проекта VIToshatrade \cite{vtrade} и по своята същност проектът е DIStributed COMPuting решение. 

\begin{figure}[h!]
  \centering
  \includegraphics[width=1.0\linewidth]{./images/pic0010.png}
  \caption{Публикуване на проект в GitHub.}
\label{fig:pic0010}
\end{figure}

Съществуват различни възможности за публикуване на програмния код, но една от най-популярните алтернативи е облачната услуга GitHub. Услугата GitHub (Фиг. \ref{fig:pic0010}) е базирана на системата за контрол на версиите Git и дава една от най-широките възможности за популяризиране на програмен код с отворен лиценз. 

\newpage
\chapter{Android активен тапет}

Технологията Android Live Wallpaper предоставя анимирани възможности за визуализация под формата на виртуален тапет. По своята същина този вид приложения не се различават драстично от останалите Android  приложения и дори могат да изпълняват почти всички техни възможности. Тъй като виртуалният тапет е постоянно активен той е идеален кандидат за реализацията на пресмятания във фонов режим. За създаването на активен тапет са необходими следните компоненти: 1. XML файл който описва компонентите на тапета; 2. Фонов модул (Android Service); 3. Подходящи флагове за достъп до ресурсите на устройството. 

\section{Манифест файл на мобилното приложение}

При създаването на приложения за операционната система Android е възприет подхода компонентите на мобилното приложение да бъдат описани в манифест файл (AndroidManifest.xml), с използването на XML синтаксис. 

\begin{figure}[h!]
  \centering
  \includegraphics[height=0.45\pdfpageheight]{./images/pic0011.png}
  \caption{Определяне на приложението като активен тапет.}
\label{fig:pic0011}
\end{figure}
\FloatBarrier

За да се определи приложението като приложение от тип активен тапет е необходимо това изрично да се маркира в манифест файла (Фиг. \ref{fig:pic0011}). Най-съществената полза от тази дефиниция е, че тя предотвратява възможността приложението да бъде инсталирано на устройства, които не поддържат възможностите за визуализация на активен тапет. 

\begin{figure}[h!]
  \centering
  \includegraphics[height=0.45\pdfpageheight]{./images/pic0012.png}
  \caption{Модул от тип "услуга".}
\label{fig:pic0012}
\end{figure}
\FloatBarrier

Когато в Android се използват много продължителни пресмятания които не са удачни за извършване в нишка е прието тези изчисления да се изнасят в модули без графичен потребителски интерфейс наречени „услуги“ (Service). Работата на активния тапет се извършва точно в такъв модул и поради тази причина в проекта е добавен един (Фиг. \ref{fig:pic0012}).

\begin{figure}[h!]
  \centering
  \includegraphics[height=0.45\pdfpageheight]{./images/pic0013.png}
  \caption{Флагове за достъп до ресурса активен тапет.}
\label{fig:pic0013}
\end{figure}
\FloatBarrier

Моделът за сигурност изисква за всяко по-особено действие да се получи изричното съгласие на потребителя. В случая с активния тапет е необходимо добавянето на android.permission.BIND\_WALLPAPER разрешение (Фиг. \ref{fig:pic0013}).

\begin{figure}[h!]
  \centering
  \includegraphics[height=0.45\pdfpageheight]{./images/pic0014.png}
  \caption{Регистрация на услугата за прослушване на съобщения от операционната система.}
\label{fig:pic0014}
\end{figure}
\FloatBarrier

Освен нуждата от разрешения за ползване на ресурсите за активен тапет е нужно услугата да се абонира за прослушване на съобщения от операционната система (Фиг. \ref{fig:pic0014}).

\begin{figure}[h!]
  \centering
  \includegraphics[height=0.45\pdfpageheight]{./images/pic0015.png}
  \caption{Референция към описателния файл на активния тапет.}
\label{fig:pic0015}
\end{figure}
\FloatBarrier

Описанието на самия активен тапет се съдържа в отделен XML файл, референция към който се посочва в манифеста (Фиг. \ref{fig:pic0015}).

\begin{figure}[h!]
  \centering
  \includegraphics[height=0.45\pdfpageheight]{./images/pic0016.png}
  \caption{Прозорец за установяване на активния тапет.}
\label{fig:pic0016}
\end{figure}
\FloatBarrier

Последният компонент в такъв тип приложение е прозорец (Android Activity) който да се стартира от операционната система и да служи за установяване на активния тапет (Фиг. \ref{fig:pic0016}). В случая се използва възможността този стартов прозорец да съчетава и функциите на прозорец с опции (Preference Activity).

\begin{figure}[h!]
  \centering
  \includegraphics[height=0.45\pdfpageheight]{./images/pic0017.png}
  \caption{XML описателен файл на тапета.}
\label{fig:pic0017}
\end{figure}
\FloatBarrier

Както вече бе споменато, активният десктоп се описва в отделен XML файл, който съдържа кратка анотация на приложението, предварителен изглед, умалена икона и името на прозореца за настройки (Фиг. \ref{fig:pic0017}). 

\section{Екран с настройки}

Тъй като активният тапет ще има и вторична задача да визуализира прогреса на пресмятанията, то е разумно към него да бъде създаден и прозорец с настройки. 

\begin{figure}[h!]
  \centering
  \includegraphics[height=0.45\pdfpageheight]{./images/pic0018.png}
  \caption{Фотографии на планината Витоша, до град София, България.}
\label{fig:pic0018}
\end{figure}
\FloatBarrier

Като визуализация е избран възможно най-опростен вариант. Няколко фотографии на планината Витоша се визуализират под формата на отрязъци с размерите на екрана който мобилното устройство има (Фиг. \ref{fig:pic0018}). В три полупрозрачни области се визуализира информация за финансовия времеви ред (код и времеви период), стълб-диаграма за входните и изходните данни и текущо състояние на изкуствената невронна мрежа (Фиг. \ref{fig:pic0019}). 

\begin{figure}[h!]
  \centering
  \includegraphics[width=1.0\linewidth]{./images/pic0019.png}
  \caption{Визуализация на информация от изчисленията.}
\label{fig:pic0019}
\end{figure}
\FloatBarrier

В настройките за активния тапет са предоставени възможности за управление на позицията и размера за трите визуални области. Също така са добавени първоначални настройки за натоварване на устройството и дали активният тапет да бъде включен (Фиг. \ref{fig:pic0020}). 

\begin{figure}[h!]
  \centering
  \includegraphics[width=1.0\linewidth]{./images/pic0020.png}
  \caption{Първоначален набор от настройки.}
\label{fig:pic0020}
\end{figure}
\FloatBarrier

Всеки прозорец в Android се описва със свой файл за разполагане (layout) и файл с Java програмен код. Описателният файл за графичен потребителски интерфейс използва XML и много наподобява съставянето на уеб страница. Когато се изготвя екран за настройки един от най-полезните инструменти в операционната система Android са споделените преференции (Shared Preferences). Те позволяват състоянието на визуалните компоненти директно да бъде запаметено в устройството под формата на ключ-стойност двойки и след това програмно тази информация да бъде използвана. 

\subsection{Описание на потребителския интерфейс под формата на XML файлове}

\begin{figure}[h!]
  \centering
  \includegraphics[height=0.45\pdfpageheight]{./images/pic0021.png}
  \caption{Визуален компонент за включване и изключване на тапета.}
\label{fig:pic0021}
\end{figure}
\FloatBarrier

На първо място е разположен визуален компонент за включване и изключване на активния тапет. Когато ключът е в състояние ON активният тапет бива стартиран, а когато е в позиция OFF активният тапет бива деактивиран (Фиг. \ref{fig:pic0021}). 

\begin{figure}[h!]
  \centering
  \includegraphics[height=0.45\pdfpageheight]{./images/pic0022.png}
  \caption{Визуален компонент за включване и изключване на тапета.}
\label{fig:pic0022}
\end{figure}
\FloatBarrier

За натоварването на устройството при извършването на фоновите пресмятания се използва списък с предварително дефинирани стойности (Фиг. \ref{fig:pic0022}). 

\begin{figure}[h!]
  \centering
  \includegraphics[height=0.45\pdfpageheight]{./images/pic0023.png}
  \caption{Стойности за натоварване на системата.}
\label{fig:pic0023}
\end{figure}
\FloatBarrier

При проектирането на системата Android една от основните цели е била максимално разделяне на графичния интерфейс от данните. Точно поради тази причина стойностите за натоварване са изнесени в отделен ресурс (Фиг. \ref{fig:pic0023}).

\begin{figure}[h!]
  \centering
  \includegraphics[height=0.45\pdfpageheight]{./images/pic0024.png}
  \caption{Позициониране на областите за визуализация.}
\label{fig:pic0024}
\end{figure}
\FloatBarrier

Позиционирането на областите за визуализация също се настройва с избор на стойности от списък (Фиг. \ref{fig:pic0024}). 

\begin{figure}[h!]
  \centering
  \includegraphics[height=0.45\pdfpageheight]{./images/pic0025.png}
  \caption{Стойности за позициониране на областите за визуализация.}
\label{fig:pic0025}
\end{figure}
\FloatBarrier

Аналогично на стойностите за натоварване, списъка с възможните позиции на областите за визуализация е изнесен в отделен ресурсен файл (Фиг. \ref{fig:pic0025}). 

\begin{figure}[h!]
  \centering
  \includegraphics[height=0.45\pdfpageheight]{./images/pic0026.png}
  \caption{Размер на областите за визуализация.}
\label{fig:pic0026}
\end{figure}
\FloatBarrier

От първоначалните характеристики последна е размерът на областите за визуализация на информацията (Фиг. \ref{fig:pic0026}). 

\begin{figure}[h!]
  \centering
  \includegraphics[height=0.45\pdfpageheight]{./images/pic0027.png}
  \caption{Стойности за размер на областите за визуализация.}
\label{fig:pic0027}
\end{figure}
\FloatBarrier

Областите за визуализация на информацията се предлагат в три размера – малък, среден и голям (Фиг. \ref{fig:pic0027}).

\subsection{Програмен код за управление на интерфейса}

Събитията предизвикани от графичния програмен интерфейс биват прихващани в специално написани за целта Java функции, така че при активирането им да бъдат извършени необходимите програмни действия. За екрана с настройки се прихващат само две събития – създаване и паузиране. 

\begin{figure}[h!]
  \centering
  \includegraphics[height=0.45\pdfpageheight]{./images/pic0028.png}
  \caption{Събитие за създаване на прозореца.}
\label{fig:pic0028}
\end{figure}
\FloatBarrier

Събитието за създаване има за цел да трансформира XML описанието на интерфейса до визуални компоненти, видими за потребителя (Фиг. \ref{fig:pic0028}). 

\begin{figure}[h!]
  \centering
  \includegraphics[height=0.45\pdfpageheight]{./images/pic0029.png}
  \caption{Събитие за пауза на прозореца.}
\label{fig:pic0029}
\end{figure}
\FloatBarrier

При събитието за пауза единствено се взема решение дали активният тапет да бъде стартиран или да бъде спрян (Фиг. \ref{fig:pic0029}). 

\newpage
\addcontentsline{toc}{chapter}{Заключение}
\chapter*{Заключение}

Основна цел на настоящото учебно помагало бе представянето на възможностите, които съвременните Android мобилни устройства могат да предложат за извършване на научни изчисления в разпределена среда\index{изчисления в разпределена среда}. Без да претендира за изчерпателност изложеният материал има за цел да провокира творческото мислене у читателя и да го вдъхнови за създаването на собствени авторски проекти с разгледаните технологии и засегнатите научни области.  

Авторите са благодарни на своите читатели за отделеното време и внимание, като горещо насърчават подаването на обратна връзка и споделянето на интересни мисли, идеи или предложения, на посочените за връзка контакти.

% Списък с използвана литература и източници на информация.
\newpage
\begin{thebibliography}{99}

\bibitem{dcinfo} Distributed Computing Info , \\\texttt{http://www.distributedcomputing.info/}

\bibitem{shuch} Paul Shuch, H. \textit{Searching for Extraterrestrial Intelligence}. Springer-Verlag Berlin Heidelberg, 2011.

\bibitem{bohn} Bohn, A., Guting, T., Mansmann, T. \textit{MoneyBee: A new product to predict stock market developments using artificial intelligence and increased calculation capacitiy} German. et al. Wirtschaftsinf, vol. 45/3, 325--333, 2003.

\bibitem{vtrade} VitoshaTrade Project , \\\texttt{https://github.com/VelbazhdSoftwareLLC/VitoshaTrade}

\end{thebibliography}

% Азбучен указател на използваните термини.
\newpage
\printindex

\end{document}
