Настоящото учебно помагало съдържа набор от примери по дисциплините „Java за напреднали“, част от магистърската програма „Софтуерни технологии в Интернет“ на Нов български университет и „Мобилни приложения“, част от баклавърската програма „Компютърни науки“ на Университета по библиотекознание и информационни технологии. Насочено е към аудитория със сериозни познания в областта на програмирането. 

Помагалото е съставено от практически примери, обхващащи цялостния цикъл за разработка на клиент-сървър базирани системи, подходящи за реализацията на изчисления в разпределена среда. Структурата на учебното помагало е така подбрана, че отделни части от него да послужат при разработването на курсови задачи и/или дипломни работи.

Свободният достъп до помагалото дава възможност то да бъде възприето и в други курсове, присъстващи в учебните планове на други учебни заведения. Изложеният материал дава възможност за разширяване на съществуващите магистърски програми, примерно в областта на изкуствения интелект.

Използваните информационни източници са предимно със справочен характер, но дават възможност на заинтересуваните читатели да разширят познанията си в засегнатите области. 

This handbook contains a set of examples on the Advanced Java course, part of the master program "Internet Software Technologies" in New Bulgarian University and Mobile Applications, part of the bachelor program "Computer Science" in the University of Library Studies and Information Technologies. It targets an audience with a strong knowledge in the field of the computer programming.

The handbook is made up of practical examples covering the entire development cycle of client-server based systems suitable for the implementation of distributed computing solutions. The structure of the handbook is selected in such way that parts of it can be used to develop course assignments and/or diploma theses.

Free access to the handbook enables it to be taken up in other courses present in the curricula of other educational institutions. The presented material allows the extension of the existing master programs, for example in the field of artificial intelligence.

The information sources used are mostly of a reference nature but allow interested readers to expand their knowledge in the areas concerned.

