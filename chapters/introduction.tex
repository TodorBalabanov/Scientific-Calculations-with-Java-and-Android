\newpage
\addcontentsline{toc}{chapter}{Предговор}
\chapter*{Предговор}

Това учебно помагало е предназначено за ученици и студенти, които биха искали да се запознаят с възможностите за реализиране на изчисления в разпределена среда\index{изчисления в разпределена среда} с използването на програмния език Java и мобилната платформа Android.

В съвременното ни ежедневие ние все по-често сме заобиколени от мобилни изчислителни устройства. Най-често това са мобилни телефони, таблети, часовници или други форми на wearables (миниатюрна електроника под формата на модни аксесоари или части от дрехи). До преди десетилетие този вид мобилни устройства бяха рядкост, а изчислителните им възможности бяха изключително ограничени. Тези два факта не позволяваха мобилните устройства да бъдат използвани за нещо повече освен основния принцип на употреба за който са създадени. С развитието на микроелектрониката и миниатюризацията на компонентите съставящи мобилните устройства техните възможности значително нараснаха за последното десетилетие. Това позволява върху този вид устройства да се извършват и допълнителни задачи, които не са били предвидени при първоначалното им проектиране. Паралелно с развитието на мобилните технологии бурен подем претърпяха и възможностите за мобилна комуникация като GSM, 3G, 4G, Wi-Fi, Bluetooth, NFC и други. Комбинацията между относително мощни мобилни изчислителни устройства и добре развита комуникационна среда открива безгранични възможности за приложение на мобилните устройства при извършването на допълнителни изчисления, в разпределена среда\index{изчисления в разпределена среда}. 

В настоящето учебно помагало ще запознаем читателите с интересните възможности, които предлагат съвременните Android мобилни устройства\index{мобилни устройства}, за постигането на резултати в научни изчислителни задачи, разпределяйки изчисленията върху физически отдалечени едно от друго устройства. Изложението на материала е организирано в следните глави. 

Глава 1 - \nameref{chapter01}: Дава кратко описание на идеята за научни изследвания и по какъв начин те могат да се впишат в ежедневния живот и използването на „умни“ мобилни устройства.

Глава 2 - \nameref{chapter02}: Представят се някои от популярните евристични подходи за решаване на сложни изчислителни задачи. Обяснява се и разликата между точните числени методи и приближените изчисления. По-сериозно внимание се отделя на генетичните алгоритми и изкуствените невронни мрежи, тъй като те са в основата на разработваната система. 

Глава 3 - \nameref{chapter03}: Излагат се различни аргументи за изготвянето на една софтуерна архитектура и се дават насоки за реалната разработка на един софтуерен проект. Изборът на архитектура за системата е насочен към трислойните модели, както в контекста на клиент-сървър система, така и в контекста на локално мобилно приложение, което се състои от потребителски интерфейс, работна логика и локална база данни. 

Глава 4 - \nameref{chapter04}: Детайлно се представя процеса по създаването на едно мобилно приложение което има за основна задача извършване на изчисления във фонов режим, чрез използване на възможностите които Android операционната система дава. Това включва приложение за активен тапет, екран за настройки и група класове за вътрешно представяне на информацията. 

Глава 5 - \nameref{chapter05}: Показва реализацията на отдалечената част от системата, а именно уеб базиран сървър с релационна база данни. За съхраняването на суровата информация за валутните котировки и параметрите на изкуствените невронни мрежи е представено MySQL базирано решение под формата на релационна база данни. За обмен на информацията между отдалечената база данни и мобилните приложения е представено PHP базирано решение. 

Глава 6 - \nameref{chapter06}: Набляга на комуникацията между отдалечения сървър и локалните клиенти, изпълнявани върху мобилни устройства. Тъй като от страната на сървъра е избрано уеб базирано решение, то в основата на комуникацията е заложен HTTP протокола и неговите възможности да предава JSON пакетирани съобщения. 

Глава 7 - \nameref{chapter07}: Разглежда обучението на изкуствени невронни мрежи, което е свързано с намиране на такива оптимални стойности за теглата, чрез които мрежата максимално добре да изпълнява задачата си. При липсата на обратни връзки, се използва обратно разпространение на грешката, а при наличието на обратни връзки - генетични алгоритми. 