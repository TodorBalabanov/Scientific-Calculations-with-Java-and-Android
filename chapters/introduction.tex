\newpage
\addcontentsline{toc}{chapter}{Foreword}
\chapter*{Preface}

This tutorial is intended for students who would like to learn about the possibilities of implementing distributed computing\index{distributed computing} using the Java programming language and the Android mobile platform.

We are increasingly surrounded by mobile computing devices in our modern everyday life. Most often, these are mobile phones, tablets, watches, or other forms of wearables (miniature electronics in the form of fashion accessories or parts of clothing). These mobile devices were rare until a decade ago, with minimal computing capabilities. These two facts prevented mobile devices from being used for anything other than the basic principle of use for which they were created. With the development of microelectronics and the miniaturization of the components that make up mobile devices, their capabilities have grown significantly over the past decade. This allows additional tasks to be performed on these devices, which were not foreseen during their initial design. Parallel to the development of mobile technologies, the opportunities for mobile communication, such as GSM, 3G, 4G, Wi-Fi, Bluetooth, NFC, and others, experienced a rapid rise. The combination of relatively powerful mobile computing devices and a well-developed communication environment opens up limitless possibilities for the application of mobile devices in performing additional computing in a distributed environment\index{distributed computing}.

In this tutorial, we introduce readers to the exciting possibilities that modern Android mobile devices\index{mobile devices} offer to achieve results in scientific computing tasks by distributing computations over physically remote devices. The presentation of the material is organized in the following chapters.

Chapter 1 - \nameref{chapter01}: Gives a brief description of scientific research and how it can fit into everyday life and using "smart" mobile devices.

Chapter 2 - \nameref{chapter02}: Some popular heuristic approaches for solving complex computational problems are presented. The difference between exact numerical methods and approximate calculations is also explained. More serious attention is paid to genetic algorithms and artificial neural networks since they are the basis of the developed system.

Chapter 3 - \nameref{chapter03}: Various arguments for the preparation of software architecture are presented, and guidelines are given for the actual development of a software project. The choice of system architecture focuses on the three-layer models, both in the context of the client-server system and in the context of a local mobile application, which consists of a user interface, work logic, and a local database.

Chapter 4 - \nameref{chapter04}: The process of creating a mobile application is presented in detail, the main task of which is to perform calculations in the background using the capabilities that the Android operating system provides. This includes a live wallpaper app, a settings screen, and a group of classes for the internal representation of information.

Chapter 5 - \nameref{chapter05}: Shows the implementation of the remote part of the system, namely a web-based server with a relational database. A MySQL-based solution in the form of a relational database is presented for storing the raw information on currency quotes and parameters of artificial neural networks. A PHP-based solution is given for exchanging data between the remote database and mobile applications.

Chapter 6 - \nameref{chapter06}: Emphasizes communication between the remote server and local clients running on mobile devices. Since a web-based solution has been chosen on the server side, the HTTP protocol and its capabilities to transmit JSON-packed messages are the basis of communication.

Chapter 7 - \nameref{chapter07}: Considers the training of artificial neural networks, which is concerned with finding such optimal values for the weights that the network performs the task it is designed for as well as possible. In the absence of feedback, backpropagation of the error is used, and genetic algorithms are used in the presence of feedback.
